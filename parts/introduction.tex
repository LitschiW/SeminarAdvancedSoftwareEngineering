\section{Introduction}
Todays programs are mostly highly customizable. Whilst buying or downloading a program a user already decides on which version of program (s)he wants to have. On Installation there are mostly multiple Options reaching from 'Language' to a selection of software features. Once a program is installed there are usually multiple startup, customization and configuration options provided for a customer. But having such a large amount of options brings some problems into the development of modern applications. We need to have a stable development strategy that can offer the creation of multiple similar products with little redundant code or components.In practice software product lines are used for this case\\
Furthermore there is the problem of performance prediction. For this we introduce some Definitions.
This paper will use the definition of \textit{feature} and \textit{configuration} as they are described in \cite{PredictingPerformanceAutmatedFeatureDetectionSiegmund2012} \inlineQuote{[...], where a feature is a stakeholder-visible behavior or characteristic of a program.} and \inlineQuote{a specific set of features, [is] called a configuration}.
The amount of possible configurations naturally lies in $\mathcal{O}(2^n)$. This scaling makes it hard to test each singular configuration for its performance or correctness. Especially if the configuration under test is unpredictably chosen by a user \cite{PredictingPerformanceAutmatedFeatureDetectionSiegmund2012}. Nevertheless this paper will only look at the non-functional performance properties of an application. Consequently efficiently analyzing a product's performance is only partially possible and behavior beyond that has to be predicted. This opens up the challenge of efficiently and accurately predicting performance. Over time a lot of methods in different disciplines have been prosed.\\
This paper aims to give a short introduction into the importance of software product lines (SPL) in regards to performance engineering and an overview over solutions for predicting and learning the performance of a configurable software system.
\TODOX{mention references}