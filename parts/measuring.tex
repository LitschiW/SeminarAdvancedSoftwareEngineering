\section{Measuring and Predicting the Performance of Highly Configurable Systems}\label{sec:measuring}

By learning about the performance difference between multiple configurations it is possible accurately predict the a programs performance. This means that one does not need ot measure all (possibly exponentially many) configurations. Instead a small sample size should be enough to predict a programs performance. Multiple ways that use different methods have been proposed over time \cite{FasterDiscoveryofFasterSystemConfigurationsSiegmund2017}. This paper will take a look at
\begin{itemize}
	\item \hyperref[sec:AFID]{Automated Feature Interaction Detection} by \citet{AutomatedFeatureDetectionSiegmund2012},
	\item an \hyperref[sec:VAPP]{incremental/statistical learning approach} by \citet{VariabilityAwarePerformancePredictionJianmeiSigmundApel},
	\item \hyperref[sec:WHAT]{\textbf{WHAT}} a spectral learning approach by \citet{FasterDiscoveryofFasterSystemConfigurationsSiegmund2017},
	\item \hyperref[sec:ProjectiveSampling]{Cost efficient sampling by \cite{CostEfficientSampling_Gou_Siegmund_2015}}
\end{itemize}

\subsection{General Approach}

\citet{VariabilityAwarePerformancePredictionJianmeiSigmundApel} define two problems that are to be solved by prediction approaches:
\begin{enumerate}
	\item Predict the performance of a not measured configurations.
	\item Find a function $f$ that shows the correlation between the properties of measured configurations and their performance value and that makes each predicted performance $f(\x)$ of a configuration $\x$ as close as possible to its actual performance.\\	
	\begin{equation}
	f : \mathcal{C} \rightarrow  \mathbb{R} \text{ such that} \sum_{(\x,y) \in S} L(y,f(\x)) \text{ is minimal}
	\end{equation}\\
	$S$ is a sample and	$L$ is a loss function to penalize errors in prediction. $(\x,y)$ is a pair of a configuration and its measured performance value.
\end{enumerate}

\noindent
There is a general pattern for the solution of those problems that comes apparent when looking at different prediction approaches. It can be devided into two steps which can be seen in \cref{fig:GeneralApproach}.\begin{figure}
	\includegraphics[page=9,clip,trim=4.1cm 5.9cm 3.9cm 19.4cm, width=\textwidth]{presentation/presentation}
	\caption{General pattern of prediction. Images from \cite{VAMOSConference}.}	\label{fig:GeneralApproach}
\end{figure}

The first step is to sample the exponential configuration space. This means finding configurations from which can be learned about the system. This is done by using effient sampling techniques like spectral sampling \cite{DistanceBasedSampling2019} or progressive sampling \cite{CostEfficientSampling_Gou_Siegmund_2015}.

Once enough and meaningful configurations are found the learning process starts. Usually the previously chosen configurations are measured, under the condition that this was not already done whilst sampling. The measurement results are fed to a learning process. A lot of different machine learning strategies can be applied \cite{VAMOSConference}. The relative papers typically use \CART's. Those will be explained further in \cref{sec:CART}.


%\subsection{Approaches to Performance Prediction}\label{sec:approachesPerformanceLearning}

Before looking at actual prediction methods one should have a look at general possibilities for prediction methods. There are 2 general approaches to predicting a component-based system's performance \cite{ComparativeanalysisAbdelaziz112011}. 

\subsubsection[Measurement Approach]{\textnormal{First there is the }measurement-based approach\textnormal{.}} A measurement-based approach uses an analysis tool to monitor an application during execution. Based on the measurements of an existing application a performance model is build and modified. This approach is highly dependent on existing software (e.g.: measured application, analysis tool, operating system). \cite{ComparativeanalysisAbdelaziz112011}

\subsubsection[Model-based Approach]{\textnormal{Secondly there is the }model-based approach\textnormal{.}}
A model-based approach relies on models created by Model Driven Development. It combines multiple models of a system to give a performance prediction. The advantage of this technique is that it does not require a system to exists. Therefore performance modelling can be done before a system is actually composed.
Automation and usage profiles are used to improve the prediction accuracy of this method. Approaches that don't use automation are generally found to be reasoning tools rather than applicable approaches. Some approaches do not consider external factors such as external service calls, usage profiles or the execution environment at all. This decreases the accuracy of their predictions. \cite{ComparativeanalysisAbdelaziz112011}

\subsubsection[Mixed Approaches]{Mixed approaches} are also possible and can occur in many different variations. For example it is possible to parametrize a model based on measured values. \cite{ComparativeanalysisAbdelaziz112011}

This paper will mainly focus on measurement based approaches.%TODO reasoning

%\citet{ComparativeanalysisAbdelaziz112011} also provide an overview over the benefits and disadvantages of each type of approach:





\subsection{Automated Feature Interaction Detection}\label{sec:AFID}

Automated feature interaction detection (\AFID) is a measurement-based approach to predicting the performance of a highly configurable system.
It was developed by \citet{AutomatedFeatureDetectionSiegmund2012}. The following section is also based on the proposing paper of \AFID~\cite{AutomatedFeatureDetectionSiegmund2012}. 

Under the usage of linear regression \AFID~tries to determine a performance influence value for each feature and feature interactions. A feature interaction is defined as a unexpected influence on the performance of a system when using a specific feature combination.
In the conducted experiments of \citet{AutomatedFeatureDetectionSiegmund2012} and \citet{CostEfficientSampling_Gou_Siegmund_2015} this method reaches average accuracies of  respectively 95\% and 85\%.
\\
\noindent
In general one can divide \AFID~into two different steps:
\begin{enumerate}
	\item Find interacting Features.
	\item Measure the performance influence of feature interactions.
\end{enumerate}
Firstly step some notation is needed. For simplicity \AFID~is defined for binary options.
\TODOX{Add overview graphic}
The composition of performance influencing units (features or feature interactions) is denoted by a $\cdot$. If two features are used simultaneously it is denoted by using a $\times$. This would also be another way to describe a configuration. So a program $P$ that uses the two features $a$ and $b$ can be denoted as $P= a \times b$.\\
If one now wants to know the performance of $P$, they have to calculate $\Pi(P)$. The exact definition of the performance influence determining function $\Pi$ can be found in the paper of \citet{AutomatedFeatureDetectionSiegmund2012}. For now it is important to note that when calculating $\Pi(P)=\Pi(a \times b)$ not only the performance influence of $a$ and $b$ are necessary but also has $a\#b$ has to be considered. The latter is the performance influence of the possible interaction between $a$ and $b$. So $\Pi(a \times b) = \Pi(a\#b) + \Pi(a) + \Pi(b)$. This can also go into higher order interaction: When using a program configuration $P_2 = a \times b \times c$ then $\Pi(P_2) = \Pi(a) +  \Pi(b) +  \Pi(c) +  \Pi(a\#b) +  \Pi(a\#c) +  \Pi(b\#c) +  \Pi(a\#b\#c)$. Some interactions do not exist or have an influence on the system, those who actually have have to be found and measured. Otherwise one could end up doing a brute-force solution again.\\\\

Finding these interactions requires to find the related interacting features themselves first.
This is done in by intelligently measuring certain configurations. \AFID~defines a features $a$ as interacting when
\begin{equation}\label{def:equation}
a \text{ interacts} \Leftrightarrow \exists C,D \subseteq \mathcal{C}| C \neq D  \land	 |\Delta a_C - \Delta a_D| \leq t
\end{equation}
with 
\begin{equation}
\begin{split}
\Delta a_C &= \Pi(C\times a) - \Pi(C)\\
&=\Pi(a\# C) + \Pi(a).
\end{split}
\end{equation}
$t$ is a threshold depending on the given performance metric.
Using these two equations one can determine whether a feature is interacting with other features using 4 measurements.
These include $\Delta a_{min} = \Pi(a \times min(a)) - \Pi(min(a))$ and $\Delta a_{max} = \Pi(a \times max(a)) - \Pi(max(a))$. $min(a)$ is a configuration that contains the minimum possible features without using $a$. Simultaneously $max(a)$ is also a configuration that contains the maximum amount of possible features without $a$. 
Once these measurements are done \cref{def:equation} can be applied with $C=\Delta a_{min}$ and $D=\Delta a_{max}$. This is done for all features to find interacting ones.
If a feature $f$ is not found to be interactive its performance influence  $\Pi(f)$ equals $\Delta f_{min}$.


Once all interacting features are found the search for the actual interactions starts. In this second step 3 different heuristics are used to determine which interactions are searched for.
 
 \newcommand{\oitem}[2]{{\item[{\parbox[t][0pt][t]{\leftmargin}{\raggedleft #1}}] {\parbox[t]{\textwidth-\leftmargin}{#2}}}}
 \begin{itemize}[leftmargin=4cm]
 	\setlength\itemsep{1em}
 	\oitem{Pair-Wise~Heuristik (PW):\label{lab:PW}}{ Most groups of interacting features appear in the size of two \cite{AutomatedFeatureDetectionSiegmund2012,AnalysisOfTheVariabilityInFortyPreprocessor_BasedSPLLiebig}. So it makes sense to look for pair interaction first.}
 	\oitem{Higher-Order Interactions Heuristic (HO):\label{lab:HO}}{
 		\citet{AutomatedFeatureDetectionSiegmund2012} only look at higher order interactions of the rank of three. Even ranks would take up to much measurement resources.
 	}
 	\oitem{Hot-Spot Features (HS):\label{lab:HS}}{
 		Based on \cite{FeatureCohesioninSPL, CanWeAvoidHighCoupling?} \citet{AutomatedFeatureDetectionSiegmund2012} assume that hot spot features exist. At last these specific type of interactions are findable too.
 	}
 \end{itemize}
Using a SAT-Solver an implication graph as seen in \autoref{fig:ImplicationTree} is generated. Each implication chain in this tree should have at least one interacting feature. When analysing the tree each chain is walked from the top down. The three heuristics will be applied in the order of PW $\rightarrow$ HO $\rightarrow$HS.  

\begin{wrapfigure}{l}{0.5\textwidth}
%\setlength\belowcaptionskip{-\baselineskip}
\includesvg[width = 0.5\textwidth]{figures/ImplicationTree}
\captionsetup{width=0.95\linewidth}
\caption{Implication tree example found in \cite{AutomatedFeatureDetectionSiegmund2012} }
\label{fig:ImplicationTree}
\end{wrapfigure}

First the influence of every feature on another chain is measured (\hyperref[lab:PW]{PW-heuristic}). In the example of \autoref{fig:ImplicationTree} the interactions would be measured in this order:\inlineQuote{$F1\#F6, F1\#F7, F4\#F6,\\ F4\#F7, F6\#F11,F7\#F11,F1\#F11,\\ F4\#F11$}\cite{AutomatedFeatureDetectionSiegmund2012}. If an interaction impact $\Delta a\#b_C$ exceeds a threshold it is recorded.

Secondly, the \hyperref[lab:HO]{higher order interaction heuristic} is applied. Higher order interactions can be relatively easily found by looking hat the results of the PW-Heuristik. Three features that interact pair-wise are likely to interact in a third order interaction. For example, looking at features $a$, $b$ and $c$- If $\Delta a\#b_{C1}$ and $\Delta b\#c_{C2}$ have been recorded $\{a\#b, b\#c, a\#c\}$ all have to be non zero to find a third order interaction. Interactions with and order higher than three are not considered to prevent too many measurements.

Lastly Hot-Spot features are detected (\hyperref[lab:HS]{HS-heuristic}). This is done by counting the interactions per feature. If the number of interactions of a feature is above a certain threshold (e.g. the arithmetic mean) it is categorized as a Hot-Spot feature. Based on the hotspot features further third order interactions are explored. Again higher order interactions are not considered to prevent too many measurements. \\
After applying the three heuristics all detected interacting features or feature combinations are assigned a $\Delta$ to represent their performance influence on the program.

\begin{wrapfigure}{r}{.5\textwidth}
	\centering
	\setlength\belowcaptionskip{-2\baselineskip}
	\captionof{table}{Results of average accurcy found by \citet{AutomatedFeatureDetectionSiegmund2012}}
	\label{tab:avgAccuracy}
	\begin{tabular}{c|c}
		Approach&avg. Accuracy\\\midrule[1pt]
		FW&79.7\%\\\hline
		PW&91\%\\\hline
		HO&93.7\%\\\hline
		HS&95.4\%\\\hline
	\end{tabular}
\end{wrapfigure}\noindent
\citet{AutomatedFeatureDetectionSiegmund2012} tested AFID on six different SPLs (Berkely DB C,Berkely DB Java, Apache, SQLite, LLVM, x264). Each program was tested under four approaches: Feature-Wise, Pair-Wise, Higher-Order, Hot-Spot (in this order). Each approach also used the data found by the previous one. Accordingly the results get better the more heuristics are used as seen in \cref{tab:avgAccuracy}. Using only the FW approach means that interactions (and the heuristics) are not considered, yet the accuracy is already at about 80\% on average. A significant improvement can be made by using the PW heuristic. It uses on average 8.5 times more measurements than the FW approach but improves the accuracy to 91\%. Using the HO or HS approach improves the accuracy further by about 2-4\%. However for Apache using the HO over the PW approach even deteriorated the average result by 3.9\% and doubled the standard variation. As already mentioned using the HS approach gives the best accuracy this is true for all 6 tested applications. \citet{AutomatedFeatureDetectionSiegmund2012} also notes that analysing SQLite only needed about 0.1\% of all possible configurations. This hints to the good scalability of AFID.

\subsection{Classification and Regression Trees}
\label{sec:CART}
The next 3 approaches all use a specific type of machine learning strategy to construct their predictors. This method is call Classification and Regression Tree's (\CART).
To create a \CART one first needs some data points. In the context of configurable software systems each point consists at least out of a configuration and an associated performance score. These data point are then fed into the algorithm seen in
 \cref{alg:CART}.
\begin{figure}
	\lstset{
		mathescape,
		breaklines=true,
	}
	\begin{lstlisting}
1. Start at the root node, add all data points to it.
2. For each option, find the set of options that minizes the sum of the node impurities in the two child nodes.
3. If a stopping criterion is reached, exit. Otherwise, apply step 2 to each child node inturn.
	\end{lstlisting}
	\captionof{lstlisting}{Pseudocode for generating a \CART. Adopted from \citet{ClassificationAndRegressionTrees}.}
	\label{alg:CART}
\end{figure}
\TODOX{beenden}

\begin{figure}
	\centering
	\includegraphics[page=4,clip,trim=3.5cm 18cm 3.5cm 1.5cm, width=\linewidth]
	{Paper/VariabilityAwarePerformancePredictionAStatisticalLearningApproach.pdf}
	\caption{Example performance model of X264 generated by CART based on the random sampling, using minimization of the sum of squared error loss \cite{VariabilityAwarePerformancePredictionJianmeiSigmundApel}.}	
	\label{fig:VAPPExampleTree}	
\end{figure}


\subsection{Variability aware Performance Prediction}\label{sec:VAPP}

The following section is based upon \citet{VariabilityAwarePerformancePredictionJianmeiSigmundApel}.\\
Variability aware Performance Prediction (\VAPP) is a statistics based approach to performance prediction. With the help of random sampling and \CART s a simple yet effective predictor can be build. In their own tests \citet{VariabilityAwarePerformancePredictionJianmeiSigmundApel} reached an average precision of 94\% whilst using a sample as large as the ones \AFID would be using under the PW heuristic. Further tests conducted by \citet{FasterDiscoveryofFasterSystemConfigurationsSiegmund2017} with the same sample size showed an accuracy of 92.4\%. 


\begin{wrapfigure}{r}{.5\linewidth}
	\vspace{-1\baselineskip}
	\setlength\belowcaptionskip{-\baselineskip}
	\includegraphics[page=3,clip,trim=11cm 13.5cm 1.5cm 10.25cm, width=\linewidth ]{Paper/VariabilityAwarePerformancePredictionAStatisticalLearningApproach}
	\caption{Overview of the Approach of Variability aware Performance Prediction \cite{VariabilityAwarePerformancePredictionJianmeiSigmundApel}}.
	\label{fig:VAPPOverview}
\end{wrapfigure}

\subsubsection[Basic Idea]{\textnormal{The} basic idea} of variability aware performance prediction can be seen in \autoref{fig:VAPPOverview}.

Two cycles can be found. 

The first cycle is outside of the dashed box and describes the basic input-output behaviour of a predictor. A user configures a new configuration $\x$ for System $A$ and asks the predictor (dashed box) for a prediction. It replies with a quantitative prediction for $\x$'s performance.

In the second cycle a actual prediction is generated based on decision rules which themselves are inturn created by simplifying a performance model (a \CART). Random sampling is used to learn the performance model.
\\\\
Like other approaches, the target of variability aware performance prediction is to get accurate predictions with only using a small sample for the creation of the performance model.

\FloatBarrier %forces the float to appear below the subsubsection
\subsubsection{Statistical Methods}\label{sec:VAPPMethods} are used to perform the actual computation. First of all a configuration is defined as an $N$-tuple $(x_1,x_2,x_3,...,x_N)$, where $N$ is the number of all available features. Each $x_i$ represents a feature and can either have the value 1 or 0 depending on whether the feature is selected or not. An actual configuration example would be $\x_j = (x_1=1,x_2=0,x_3=1,\dots,x_N = 1)$. All valid configurations of a system are denoted as $\X$.\\

\VAPP~uses the tupel definition of a configuration. It further defines that for each configuration of $\mathcal{C}$ an actual performance value $y_j$ can be assigned. $\Y$ denotes the performance of all configurations of a system.\\
For formal correctness it is assumed that each option of a configuration actually influences the performance of the system. Otherwise a \CART could not be applied.

Combining $\Y_\X$ with $\X_\mathcal{C}\subset\mathcal{C}$ gives a sample $S$. $\Y_\X$ are the to $\X_C$ associated and measured performance values. Now the two problems arise that \VAPP tries to solve:
\begin{enumerate}
	\item Predict the performance of the not measured configurations $\hat{=}\;\X\backslash\X_\mathcal{C}$.
	\item Find a function $f$ that shows the correlation between $\X_\mathcal{C}$ and $Y_\X$ and that makes each predicted performance $f(\x)$ of $\x$ as close as possible to its actual performance.\\	
	\begin{equation}
	f : \mathcal{C} \rightarrow  \mathbb{R} \text{ such that} \sum_{\x,y \in S} L(y,f(\x)) \text{ is minimal}
	\end{equation}\\
	 $L$ is a loss function to penalize errors in prediction.
\end{enumerate}
This is done with the help of CART. All sample configurations get categorized into a binary trees leafs. A configurations selection of features determines its location in the tree. The distribution of samples inside the tree is determined with the goal of minimizing the total prediction errors per segment (sub-trees). An example tree can be found in \cref{fig:VAPPExampleTree}.
For each leaf one can determine the \textit{local model} $\ell$
\begin{equation}
	\ell_{S_i} = \frac{1}{|S_i|} \sum_{y_j \in S_i} y_j
\end{equation}
As a loss function to penalize the prediction errors \citet{VariabilityAwarePerformancePredictionJianmeiSigmundApel} choose the sum of squared error loss:
\begin{equation}
	\sum_{y_j \in S_i} L(y_i,\ell_{S_i}) = \sum_{y_j \in S_i} (y_j - \ell_{S_i})^2
\end{equation}
Therefore the best split for a segment $S_i$ is found when
\begin{equation*}
\sum_{y_j \in S_{iL}} L(y_i,\ell_{S_{iL}}) + \sum_{y_j \in S_{iR}} L(y_i,\ell_{S_{iR}})
\end{equation*}
is minimal. To prevent \textit{under}- or \textit{overfitting}\cite{ElementsOfStatisticalLearning} the recursive splitting has to be stopped at the right time. This is possible by manual parameter tuning or using a empirical-determined automatic terminator. \\
Now to the actual calculation of the quantitative prediction. Assuming there are $q$ leafs in our tree than $f(\mathrm{x})$ is defined as:
\begin{equation}\textsl{}
f(\mathrm{x})=\sum_{i=1}^{q} \ell_{S_i}I(\mathrm{x}\in S_i)
\end{equation}
where $I(\mathrm{x}\in S_i)$ is an indicator function to indicates that $\mathrm{x}$ belongs to a leaf $S_i$.\\
For the example of \autoref{fig:VAPPExampleTree} $f(\mathrm{x})$ unwraps to:
\begin{align*}
f(x) = 255&* I(x_{14}=1,x_7=0)\\[-0.1cm]
	 + 268&* I(x_{14}=1,x_7=1)\\[-0.1cm]
	 + 402&* I(x_{14}=0,x_{15}=1,x_3=0)\\[-0.1cm]
	 + 508&* I(x_{14}=0,x_{15}=1,x_3=1)\\[-0.1cm]
	 + 571&* I(x_{14}=0,x_{15}=0,x_3=1)\\[-0.1cm]
	 + 626&* I(x_{14}=0,x_{15}=0,x_3=0)
\end{align*}
Every possible configuration $\mathrm{x}$ is associated with a leaf of the tree. Therefore $f(\mathrm{x})$ can always be applied.\\
For their Experiment \citet{VariabilityAwarePerformancePredictionJianmeiSigmundApel} test the same software systems as \citet{AutomatedFeatureDetectionSiegmund2012} (\cref{sec:AFID}). They also compare their prediction results with the results produced by SPLConquerer under \AFID.\\
Unlike \AFID~the size of a sample for variability aware performance prediction can be chosen freely. \citet{VariabilityAwarePerformancePredictionJianmeiSigmundApel} use 4 different sample sizes based on the size of the program. For a program with $N$ features they use samples the size of $N,2N,3N \text{ and } M$. $M$ is the amount of configurations measured by SPLConquerer's using the \hyperref[lab:PW]{PW heuristic}.
It is found that the prediction accuracy increases linear with the size of the sample. It is also found that for using a small sample with the size of $N$ the prediction accuracy was at 92\%. However for Berkeley DB (C) the prediction rate with $N$ sized samples was at 112.4$\pm$354.6\%\footnote{$\pm$354.6 indicates the standard deviation}. This results into an average accuracy of only 28.6$\pm$68.9\%. Using a sample size of $M$ significantly improves the average prediction accuracy to 93.8\%.\\
Further \citet{VariabilityAwarePerformancePredictionJianmeiSigmundApel} comparer their approach with \AFID. This can be done since $N$ also equals the amount of configurations measured by the \hyperref[lab:FW]{FW heuristic}.\\
As already established variability aware performance prediction is not accurate for small sample sizes so it is no surprise that \AFID~with the \hyperref[lab:FW]{FW heuristic} performace better at $20.3\pm21.2$\% with a sample size of $N$. However, when using samples of size $M$ SPLConquerer's \hyperref[lab:PW]{PW heuristic} only reaches an average precision of 90.9\% compared to the already mentioned 93.9\% of \citet{VariabilityAwarePerformancePredictionJianmeiSigmundApel}'s approach.
Using the \hyperref[lab:HO]{HO} or \hyperref[lab:HS]{HS heuristic} of \AFID can produce a precision of up to 95\% but requires more measurements. This is not covered by \cite{VariabilityAwarePerformancePredictionJianmeiSigmundApel}.
%TODO vllt selbst prüfen?

\subsection{WHAT}\label{sec:WHAT}


%The test also show that for a low number of features a Brute-Force (BF) approach might also be viable. Measuring all configurations of Apache took about 213h where as the HS approach took 159h. BF guarantees a 100\% correct prediction where as the HS approach only had 94.7\% accuracy. Its is also worth noting that these tests were done on computers that are (from today's point of view) fairly slow \cite{CPUDatabase}. 

%Performance problems occuring after a while are not covered or predictable by these solutions. They go back to the old Holding problem