\section{Definitions}

Before the actual approaches are discussed it is important to pinpoint the definitions of terms which are used in this paper.

This paper often uses the terms \inlineQuote{parameters}, (configuration) \inlineQuote{options} or \inlineQuote{features}. These terms are all equivalent and describe ways to adjust and optimize functional and non-functional properties of a software system \cite{DistanceBasedSampling2019}.
One can divide into different types of options. Binary options usually have a value of 0 or 1 and describe the activation of a feature. Non-binary options support a wider range of values. For example this could be a setting for the stack-size allowed for a program. Non-numeric options support the input of text. Those could be paths or other addresses.
The set of all configuration options is denoted as $\mathcal{O}$.

A \textit{configuration} can be defined in multiple different ways. \citet{DistanceBasedSampling2019} define a configuration as a function $c : \mathcal{O} \rightarrow \{0,1\}$. It assigns a 1 to each element of $\mathcal{O}$ that is selected and a 0 to those which are not used. \citet{VariabilityAwarePerformancePredictionJianmeiSigmundApel} and
\citet{FasterDiscoveryofFasterSystemConfigurationsSiegmund2017} have a similar approach. But instead of using a function to describe $c$ they use a vector or an n-tuple over $\mathbb{Z}^+_2(= \{0,1\})$. Each position of those enumerations is associated with exactly one feature. As in \citet{DistanceBasedSampling2019} a 1 indicates an activation of a feature and a 0 means that the feature is not used. These definitions obviously describes binary options only, but can be expanded to support non-binary options by using the co-domain of $\mathbb{N}_0$ instead of $\{0,1\}$.

The \textit{configuration space} describes all valid configurations of a system. It is denoted as $\mathcal{C}$.

A \textit{sample} is the subset of a \textit{configuration space} that contains all configurations that are complied and measured during the process of sampling and learning. It may also contain the measured performance scores for each configuration.

To describe the quality of an approach an \textit{accuracy} metric is often used. The \textit{accuracy} is defined as 1-\textit{fault rate}. And in turn the fault rate (or error rate) is defined as 
\begin{equation}
	\text{fault rate}= \frac{|actual-predicted|}{actual}.
\end{equation} This definition can be found in \cite{FasterDiscoveryofFasterSystemConfigurationsSiegmund2017} and \cite{AutomatedFeatureDetectionSiegmund2012}.